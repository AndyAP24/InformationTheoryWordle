\documentclass{article}
\usepackage[utf8]{inputenc}
\usepackage[a4paper, total={6in, 8in}]{geometry}

\title{Solving Wordle using Information Theory: Tasks 1-6}
\author{Andy Arrigony Pérez, Maryam Abuissa, Daniel Flores García}
\date{March 2023}


\begin{document}

\maketitle

\section{Task 1}

Consider $I[T; G] = H(T) - H(T|G)$. But note that $H(T) = H(T|G)$ because $T$ and $G$ are independent. So $I[T; G] = 0$.

\section{Task 2}

Consider $I[T;G|P] = H(T|P) - H(T|G,P)$. But note that $H(T|P) > H(T|G,P)$, intuitively because knowing both the guess and the pattern leads to less uncertainty about $T$ than knowing just the pattern.

\section{Task 3}

If we know both the guess and the target, we can deduce the pattern. Hence, $H[P | G, T] = 0$.

\section{Task 4}

Consider: $I[T; P|G] = H(T|G) - H(T|P,G)$. But note that $H(T|G) = H(T) - I[T;G] = H(T)$, which does not depend on $\pi$. Thus, we need to minimize the subtractive term $H(T|P,G)$ (since it \emph{does} depend on $\pi$) in order to maximize $I[T; P|G]$.

\section{Task 5}

Let $R$ be the alphabet of $P$. Consider:

\[
  H[T|G,P] = \sum_{t \in W} \sum_{g \in W} \sum_{r \in R} p(t, g, r)\log (\frac{1}{p(t|g,r)}) = \sum_{t \in W} \sum_{g \in W} \sum_{r \in R} p(r | t,g)p(t,g) \log(\frac{1}{p(t|g,r)})
\]

But note that $\exists r' \in R$ s.t. $p(r'  |t,g) = 1$ and therefore $p(r^* |t,g) = 0 \, \,\forall r^* \in R$ s.t. $r* \neq r'$. Therefore:

\[
    = \sum_{t \in W} \sum_{g \in W} p(t,g) \log (\frac{1}{p(t|g,r')})
\]

Moreover, by uniformity of $T$, $p(t|g,r') = \frac{1}{|W'(g,t)|}$, where $W'(g,t)$ is a pruned alphabet. Moreover, changing the summation order:

\[
    = \sum_{g \in W} \sum_{t \in W} p(t,g) \log(|W'(g,t)|) = \sum_{g \in W} \sum_{t \in W} p(t)p(g) \log(|W'(g,t)|) = \frac{1}{|W|} \sum_{g \in W} \pi(g) \sum_{t \in W}  \log(|W'(g,t)|)
\]

And letting $\alpha_g = \sum_{t \in W}  \log(|W'(g,t)|)$:

\[
    = \frac{1}{|W|}\sum_{g \in W} \pi(g) \alpha_g   
\]

\section{Task 6}

Suppose there are multiple guesses $g_1, g_2, ..., g_k \in W$ with the same minimum value for $\alpha_{g_1} = \alpha_{g_2} = ... = \alpha_{g_k}$. Then, let $\pi$ be a distribution s.t. $\sum_{i =1}^{k} \pi(g_i) = 1$ and $\pi(g_i) \geq 0$. Note that $\pi(g_j) = 0 \,\,\, \forall g_j \in W$ s.t.  $g_j \neq g_i$. Additionally, let $\alpha_{\phi} = \alpha_{g_i}$ for any $i \in [1,..,k].$ Consider:

\[
    \frac{1}{|W|}\sum_{g \in W} \pi(g) \alpha_g = \frac{1}{|W|}\sum_{i = 1}^{k} \pi(g_i) \alpha_\phi = \frac{\alpha_\phi}{|W|}\sum_{i=1}^{k}\pi(g_i) 
\]
\[
    = \frac{\alpha_\phi}{|W|}
\]

Now, to show that $\pi$ is optimal, we show that for an arbitrary distribution $\pi^*$, $ \frac{1}{|W|}\sum_{g \in W} \pi^*(g) \alpha_g \geq  \frac{1}{|W|}\sum_{g \in W} \pi(g) \alpha_g$. Let $g_j$ be a guess such that $g_j \neq g_i \,\,\, \forall  i \in [1, ..., k], \forall j \in [1, ..., t]$. Consider:



\[
    \frac{1}{|W|}\sum_{g \in W} \pi^*(g) = \frac{1}{|W|}(\sum_{i = 1}^{k} \pi^*(g_i) \alpha_\phi + \sum_{j = 1}^{t} \pi^*(g_j)\alpha_j)
\]

But note that: 

\[
    \sum_{j = 1}^{t} \pi^*(g_j)\alpha_j = \sum_{j = 1}^{t} \pi^*(g_j) (\alpha_\phi + \beta_j)
\]

for some $\beta_j \geq 0$, because $\alpha_\phi$ is minimum and $\geq 0$. So:

\[
    = \sum_{j = 1}^{t} \pi^*(g_j) \alpha_\phi + \sum_{j = 1}^{t} \pi^*(g_j) \beta_j
\]

Plugging this into the earlier expression: 

\[
    \frac{1}{|W|}(\sum_{i = 1}^{k} \pi^*(g_i) \alpha_\phi + \sum_{j = 1}^{t} \pi^*(g_j)\alpha_j) = \frac{1}{|W|}(\sum_{i = 1}^{k} \pi^*(g_i) \alpha_\phi + \sum_{j = 1}^{t} \pi^*(g_j) \alpha_\phi + \sum_{j = 1}^{t} \pi^*(g_j) \beta_j)
\]

\[
    = \frac{1}{|W|}(\alpha_\phi (\sum_{i = 1}^{k} \pi^*(g_i)  + \sum_{j = 1}^{t} \pi^*(g_j)) + \sum_{j = 1}^{t} \pi^*(g_j) \beta_j) = \frac{1}{|W|}(\alpha_\phi + \sum_{j = 1}^{t} \pi^*(g_j) \beta_j)
\]

Because $\sum_{j = 1}^{t} \pi^*(g_j) \geq 0$:

\[
    \geq \frac{\alpha_\phi}{|W|}
\]

\end{document}
